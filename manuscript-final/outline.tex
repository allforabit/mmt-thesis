% Created 2017-08-04 Fri 20:38
% Intended LaTeX compiler: pdflatex
\documentclass[a4paper,12pt]{article}
\usepackage[utf8]{inputenc}
\usepackage[T1]{fontenc}
\usepackage{graphicx}
\usepackage{grffile}
\usepackage{longtable}
\usepackage{wrapfig}
\usepackage{rotating}
\usepackage[normalem]{ulem}
\usepackage{amsmath}
\usepackage{textcomp}
\usepackage{amssymb}
\usepackage{capt-of}
\usepackage{hyperref}
\usepackage[a4paper, total={150mm,237mm}, left=30mm, top=30mm,]{geometry}
\author{Kevin Nolan}
\date{\today}
\title{Kevin Nolan MMT thesis outline}
\hypersetup{
 pdfauthor={Kevin Nolan},
 pdftitle={Kevin Nolan MMT thesis outline},
 pdfkeywords={},
 pdfsubject={Sketching sounds in the modern web browser},
 pdfcreator={Emacs 25.0.94.2 (Org mode 9.0.9)}, 
 pdflang={English}}
\begin{document}

\maketitle
\tableofcontents


\section{Introduction}
\label{sec:orgcfcd553}

\section{Background to study}
\label{sec:org97ad61c}
\subsection{Introduction}
\label{sec:org6a91cc1}
\begin{itemize}
\item Introduce what will be discussed in this section
\item A brief historical perspective
\end{itemize}
\subsection{The idea generation stage of music production}
\label{sec:orgee698d6}
\begin{description}
\item[{Introduction}] Give an overview of different stages of music production
as outlined in \cite{duignan_computer_2008} and end with
focus on the initial idea generation
\item Using an instrument
\item Scoring software
\item Quirky music tools such as "Loopy"
\item Most commonly in electronic music, DAWS
\end{description}
\subsection{Discussion of DAW Metaphors}
\label{sec:orgc1d4147}
\begin{enumerate}
\item Introduction
\item Piano roll
\item Mixing console
\item Plugins
\end{enumerate}
\subsection{Discussion of more open systems}
\label{sec:org183e1d4}
\begin{enumerate}
\item Introduction
\item Textual programming systems
\begin{itemize}
\item Supercollider
\item Csound
\item Commonmusic
\item Nyquist
\item Chuck
\end{itemize}
\item Graphical programming systems
\begin{itemize}
\item Max MSP
\item Pure data
\item Ircam Open music
\item Plogue Bidule
\item Jeskola Buzz
\end{itemize}
\end{enumerate}
\subsection{Music in the browser: a new frontier}
\label{sec:org582f749}
Discuss drum machines, etc available on the web. Note the accessibility that
they provide.
\subsection{Conclusion}
\label{sec:orge35fa60}
While the dominant metaphors used in DAWs have their uses they can lead to
limitations in the creative process particularly at the early stage of ideas
creation. More open system give too much power and impede the creative process.


\section{Similar work}
\label{sec:org21aa393}
\subsection{Introduction}
\label{sec:orge91cf1d}
\subsection{Sonic sketching from a historical perspective}
\label{sec:org586cff9}
\begin{enumerate}
\item Visual music
\begin{enumerate}
\item Oskar Fischinger
\item Fantasia
\end{enumerate}
\item Legacy synth systems
\begin{enumerate}
\item Oramics
\item UPIC
\item ANS synth
\end{enumerate}
\end{enumerate}
\subsection{Sonic sketching in the twenty first century}
\label{sec:orgd8b677b}
\begin{enumerate}
\item Golan Levin \cite{levin_painterly_2000}
\item Music animation machine \cite{malinowski_music_2017}
\item SonicPainter by William Coleman \cite{coleman_sonicpainter:_2015}
\item Fischinger google doodle
\end{enumerate}
\subsection{Alternative music systems on the web}
\label{sec:org4582e4e}
(Aphex Twin link)
\subsection{Summary of currently available music creation systems}
\label{sec:org764cfbe}
(Note that UPIC style sketch synths are not available online)
\subsection{Conclusion}
\label{sec:org9011e2c}
Both the twentieth and twenty first century have seen a great deal of
experimentation with ideas of visualizing music and sketching. The audio
processing capabilities available in modern browsers offers an opportunity to
explore and refine less mainstream music creation metaphors.


\section{My approach}
\label{sec:org234a3e0}
\subsection{Introduction}
\label{sec:orgf8dcfea}
\subsection{Appraisal of current options}
\label{sec:org147952e}
\begin{itemize}
\item Availability
\item Usage style - instrument like (Levin)
\end{itemize}
\subsection{Approach - theory}
\label{sec:orgcca6374}
\begin{enumerate}
\item HCI considerations, in particular NUI \cite{wigdor_brave_2011}
\item The Musical Interface Technology Design Space \cite{overholt_musical_2009}
\end{enumerate}
\subsection{Approach - practice}
\label{sec:orgea92578}
\begin{enumerate}
\item Introduction
\item Delivery on Web Browser
\begin{enumerate}
\item Modern Web Browser as a delivery platform :: discuss pros and cons and situations
where it is likely to be a good option. I.e. prototyping where feedback
is important. Disadvantages performance, can't be used with pro audio
software such as ASIO. \cite{adenot_web_2017}
\item Benefits of using Tone.js \cite{mann_interactive_2015}
\item Paper.js for the graphics system
\begin{itemize}
\item Scenegraph
\item Line smoothing
\item Vector system
\end{itemize}
\end{enumerate}
\item FM synthesis :: Give a brief overview of FM synthesis and why it was a
good choice for the application
\item Live coding workflow
\begin{enumerate}
\item Introduction
\begin{itemize}
\item The morphic interface
\item Mention precedents such as smalltalk squeek
\end{itemize}
\item React.js framework to allow for a declaritive programming model as well
as a live code reloading workflow
\item Clojurescript
\begin{enumerate}
\item Relationship to clojure
\item Benefits of using clojurescript
\begin{enumerate}
\item Immutable data structures (Binary tree)
\item Functional programming paradigm
\item Live code reloading (particularly when used in conjunction with
react.js)
\end{enumerate}
\end{enumerate}
\item Managing state with Re-frame
\begin{itemize}
\item Describe programming model
\item It's relationship to FRP
\end{itemize}
\end{enumerate}
\end{enumerate}
\subsection{Conclusion}
\label{sec:orga2b93c0}

\section{Execution}
\label{sec:org24bf36e}

\subsection{Introduction}
\label{sec:org2c235eb}
\subsection{Early prototype work}
\label{sec:org4a6019c}
\begin{enumerate}
\item Melodypainter
\item SonicSketch shape version
\item Porting William Coleman's SonicPainter
\end{enumerate}
\subsection{Actual implementation}
\label{sec:orgffe94eb}
\subsubsection{Setting up the architecture}
\label{sec:orgcc20f04}
\begin{enumerate}
\item Clojurescript and javascript npm modules
\item Paper.js and react.js (paper.js bindings)
\item Tone.js and react.js
\item Reagent and react.js paper.js bindings
\end{enumerate}
\subsubsection{Core functionality - timeline events (or notes)}
\label{sec:orgf2c29ee}
\begin{enumerate}
\item Introduction
\begin{itemize}
\item Describe the core functionality
\item Describe core entities
\end{itemize}
\item Add timeline event
\begin{itemize}
\item Business logic
\item UI
\item Audio
\end{itemize}
\item Add vibrato
\begin{itemize}
\item Business logic
\item UI
\item Audio
\end{itemize}
\item Remove note
\item Move note
\item Change sound (preset system)
\end{enumerate}

\subsubsection{Secondary functionality}
\label{sec:org89e7f42}
\begin{enumerate}
\item Introduction
\item Transport controls
\item Animation (current play position \& notes)
\item Undo and redo
\item Fullscreen
\item Outer UI
\item Save and load file
\end{enumerate}

\subsection{Conclusion}
\label{sec:org01905a0}
\begin{itemize}
\item Summarise the resulting artifact
\end{itemize}

\section{Evaluation}
\label{sec:orgc7754e3}
\subsection{Introduction}
\label{sec:orgb4c782f}
\subsection{Initial pilot test}
\label{sec:orgfd63fde}
\begin{description}
\item[{Introduction}] Describe methodology used
\item Describe results
\item Describe comments and feedback
\item Verified that it was approachable and basically worked as a NUI application
\end{description}
\subsection{Exhibition}
\label{sec:org5f9f7f4}
\subsection{Conclusion}
\label{sec:orge950425}

\section{Conclusion and further work}
\label{sec:org733257a}
\subsection{Summary of work completed}
\label{sec:org49fd10e}
\subsection{Broader implications of development}
\label{sec:orgfdd514d}
\begin{itemize}
\item Incorporate ideas into DAWs (move away from tracks and mixers)
\item Methodology used very successful for prototyping (delivery on web platform,
live code reloading)
\end{itemize}
\subsection{Future work}
\label{sec:orgb80f054}
\subsubsection{Performance improvements}
\label{sec:orgf04e15a}
\begin{itemize}
\item Try different audio engines, in particular wasm based
\item Move graphics to GPU based system
\end{itemize}
\subsubsection{Broaden visual language}
\label{sec:org6673bbe}
\begin{itemize}
\item Incorporate and visualize more control data
\end{itemize}
\subsubsection{Allow for larger structures}
\label{sec:org0b70dad}
\begin{itemize}
\item Perhaps by scrolling
\item Or multiple canvases/scenes
\item "Smart brushes" that would draw arpeggios, or similar generative structures
\end{itemize}

\section{References}
\label{sec:org19dcd36}
\end{document}